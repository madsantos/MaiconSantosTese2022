\documentclass[tese,capa]{texufpel}

\usepackage{madsantos}

\unidade{Centro de Desenvolvimento Tecnológico}
\programa{Programa de Pós-Gradua\-ção em Computação}
\curso{Ciência da Computação}

\unidadeeng{Technology Development Center}
\programaeng{Postgraduate Program in Computing}
\cursoeng{Computer Science}

\title{Impactos da Adoção de Infraestruturas de Nuvem no Desenvolvimento de Pesquisas Acadêmicas}

\author{Santos}{Maicon Ança dos}
\advisor[Prof.~Dr.]{Cavalheiro}{Gerson Geraldo H.}

%Palavras-chave em PT_BR
\keyword{Palavrachave-um}
\keyword{Palavrachave-dois}
\keyword{Palavrachave-tres}
\keyword{Palavrachave-quatro}

%Palavras-chave em EN_US
\keywordeng{Keyword-one}
\keywordeng{Keyword-two}
\keywordeng{Keyword-three}
\keywordeng{Keyword-four}

\begin{document}

\maketitle 

\sloppy

\fichacatalografica

%Composição da Banca Examinadora
\begin{aprovacao}{30 de fevereiro de 2019} %data da banca por extenso
\noindent Prof. Dr. Marilton Sanchotene de Aguiar (orientador)\\
Doutor em Computação pela Universidade Federal do Rio Grande do Sul.\\[1cm]

\noindent Prof. Dr. Paulo Roberto Ferreira Jr.\\
Doutor em Computação pela Universidade Federal do Rio Grande do Sul.\\[1cm]

\noindent Prof. Dr. Ricardo Matsumura Araujo\\
Doutor em Computação pela Universidade Federal do Rio Grande do Sul.\\[1cm]

\noindent Prof. Dr. Luciano da Silva Pinto\\
Doutor em Biotecnologia pela Universidade Federal de Pelotas.
\end{aprovacao}

%Opcional
\begin{dedicatoria}
  Dedico\ldots 
\end{dedicatoria}

%Opcional
\begin{agradecimentos}
  Agradeço\ldots 
\end{agradecimentos}

%Opcional
\begin{epigrafe}
  Só sei que nada sei.\\
  {\sc --- Sócrates}
\end{epigrafe}

%Resumo em Portugues (no maximo 500 palavras)
\begin{abstract}
Bla blabla blablabla bla.  Bla blabla blablabla bla.  Bla blabla
blablabla bla.  Bla blabla blablabla bla.  Bla blabla blablabla bla.
Bla blabla blablabla bla.  Bla blabla blablabla bla.  Bla blabla
blablabla bla.  Bla blabla blablabla bla.  Bla blabla blablabla bla.
Bla blabla blablabla bla.  Bla blabla blablabla bla.  Bla blabla
blablabla bla.  Bla blabla blablabla bla.  Bla blabla blablabla bla.
Bla blabla blablabla bla.  Bla blabla blablabla bla.  Bla blabla
blablabla bla.  Bla blabla blablabla bla.  Bla blabla blablabla bla.
Bla blabla blablabla bla.
\end{abstract}

%Resumo em Inglês (no maximo 500 palavras)
\begin{englishabstract}{Impacts of the Adoption of Cloud Infrastructures on the Development of Academic Research}
Bla blabla blablabla bla.  Bla blabla blablabla bla.  Bla blabla
blablabla bla.  Bla blabla blablabla bla.  Bla blabla blablabla bla.
Bla blabla blablabla bla.  Bla blabla blablabla bla.  Bla blabla
blablabla bla.  Bla blabla blablabla bla.  Bla blabla blablabla bla.
Bla blabla blablabla bla.  Bla blabla blablabla bla.  Bla blabla
blablabla bla.  Bla blabla blablabla bla.  Bla blabla blablabla bla.
Bla blabla blablabla bla.  Bla blabla blablabla bla.  Bla blabla
blablabla bla.  Bla blabla blablabla bla.  Bla blabla blablabla bla.
Bla blabla blablabla bla.
\end{englishabstract}

%Lista de Figuras
\listoffigures

%Lista de Tabelas
\listoftables

%lista de abreviaturas e siglas
\begin{listofabbrv}{ABNT}%coloque aqui a maior sigla para ajustar a distância
        \item[ABNT] Associação Brasileira de Normas Técnicas
        \item[NUMA] Non-Uniform Memory Access
        \item[SIMD] Single Instruction Multiple Data
        \item[SMP] Symmetric Multi-Processor
        \item[SPMD] Single Program Multiple Data
\end{listofabbrv}

%Sumario
\tableofcontents

\chapter*{Especificação do Caso de Estudo}

Anotações:
  \begin{itemize}
      \item Apresentar extensões para o CloudSim é uma contribuição
      \item Queremos contabilizar o custo de uma infraestrutura em nuvem:
        \begin{itemize}
            \item Considerando um trabalho atendido somente com os recursos locais
            \item Considerando o atendimento com o cluster "federado"
            \item Considerando o cloud bursting
        \end{itemize}
        \item Os cenários devem ser simulados de forma a permitir contabilizar estes custos.
  \end{itemize}

\textbf{Passo 1: Implantação do ambiente de nuvem}\\
- identificar o "nó computacional" base (vai ser definido da cabeça)\\
- identificar as instituições parceiras\\
- identificar quantos nós cada instituição tem\\
- interligar as instituições\\

\textbf{Passo 2: avaliação preliminar}\\
- gerar carga a partir das instituições parceiras\\
-- em função de traço sintético\\
- utilizar uma estratégia de distribuição da carga "simples" (algo que considere apenas disponibilidade de recursos nas outras instituições)\\
- avaliar o comportamento a medida em que escala o aumento da demanda de processamento por instituição ou globalmente\\
- individualizar o "custo" de uso da nuvem pelas instituições\\
-- contabilizar o custo de ociosidade\\

\textbf{Passo 3: Elaboração de estratégia de escalonamento e identificação dos limites da estrutura}\\
- Propor uma estratégia de escalonamento mais elaborada, considerando alguns critérios (temos que ver estes critérios)\\
- Reavaliar os custos\\

\textbf{Passo 4: determinar vantagens e desvantagens (financeiras) de realizar cloud bursting}\\
- Agora se apresenta a necessidade de "crescer" a infraestrutura: aumentar a quantidade de recursos ou alugar uma nuvem pública\\
- Reaplicar a estratégia do Passo 3 neste novo contexto\\
- Reavaliar os custos\\

\chapter{Introdução}

\section{Uma subseção}
Bla blabla blablabla bla.  Bla blabla blablabla bla.  Bla blabla
blablabla bla.  Bla blabla blablabla bla.  Bla blabla blablabla bla.
Bla blabla blablabla bla.  Bla blabla blablabla bla.  Bla blabla
blablabla bla.  Bla blabla blablabla bla.  Bla blabla blablabla bla.
Bla blabla blablabla bla.  Bla blabla blablabla bla.  Bla blabla
blablabla bla.  Bla blabla blablabla bla.  Bla blabla blablabla bla.
Bla blabla blablabla bla.  Bla blabla blablabla bla.  Bla blabla
blablabla bla.  Bla blabla blablabla bla.  Bla blabla blablabla bla.
Bla blabla blablabla bla.

Bla blabla blablabla bla.  Bla blabla blablabla bla.  Bla blabla
blablabla bla.  Bla blabla blablabla bla.  Bla blabla blablabla bla.
Bla blabla blablabla bla.  Bla blabla blablabla bla.  Bla blabla
blablabla bla.  Bla blabla blablabla bla.  Bla blabla blablabla bla.
Bla blabla blablabla bla.  Bla blabla blablabla bla.  Bla blabla
blablabla bla.  Bla blabla blablabla bla.  Bla blabla blablabla bla.
Bla blabla blablabla bla.  Bla blabla blablabla bla.  Bla blabla
blablabla bla.  Bla blabla blablabla bla.  Bla blabla blablabla bla.
Bla blabla blablabla bla~\citet{Moore:1979:MAI,Aguiar:2005}.

Bla blabla blablabla bla.  Bla blabla blablabla bla.  Bla blabla
blablabla bla.  Bla blabla blablabla bla.  Bla blabla blablabla bla.
Bla blabla blablabla bla.  Bla blabla blablabla bla.  Bla blabla
blablabla bla.  Bla blabla blablabla bla.  Bla blabla blablabla bla.
Bla blabla blablabla bla.  Bla blabla blablabla bla.  Bla blabla
blablabla bla.  Bla blabla blablabla bla.  Bla blabla blablabla bla.
Bla blabla blablabla bla.  Bla blabla blablabla bla.  Bla blabla
blablabla bla.  Bla blabla blablabla bla.  Bla blabla blablabla bla.
Bla blabla blablabla bla~\cite{vonNeumann:1966:TSR}.

\section{Outra seção}

Bla blabla blablabla bla.  Bla blabla blablabla bla.  Bla blabla
blablabla bla.  Bla blabla blablabla bla.  Bla blabla blablabla bla.
Bla blabla blablabla bla.  Bla blabla blablabla bla.  Bla blabla
blablabla bla.  Bla blabla blablabla bla.  Bla blabla blablabla bla.
Bla blabla blablabla bla.  Bla blabla blablabla bla.  Bla blabla
blablabla bla.  Bla blabla blablabla bla.  Bla blabla blablabla bla.
Bla blabla blablabla bla.  Bla blabla blablabla bla.  Bla blabla
blablabla bla.  Bla blabla blablabla bla.  Bla blabla blablabla bla.
Bla blabla blablabla bla~\ref{tabela}.

\begin{table}
  \begin{center}
    \caption{Nome da Tabela}\label{tabela}
    \begin{tabular}{p{4cm}p{5cm}p{6cm}}
      \hline
      Blabla & Blabla & Blablabla\\
      \hline
      {\small Bla} & {\small Blabla} & {\small\em Bla blabla blablabla blabla
        blablabla blabla blablabla.}\\
      {\small Bla} & {\small Blabla} & {\small\em Bla blabla blablabla blabla
        blablabla blabla blablabla.}\\
      {\small Bla} & {\small Blabla} & {\small\em Bla blabla blablabla blabla
        blablabla blabla blablabla.}\\
      {\small Bla} & {\small Blabla} & {\small\em Bla blabla blablabla blabla
        blablabla blabla blablabla.}\\
      {\small Bla} & {\small Blabla} & {\small\em Bla blabla blablabla blabla
        blablabla blabla blablabla.}\\
      {\small Bla} & {\small Blabla} & {\small\em Bla blabla blablabla blabla
        blablabla blabla blablabla.}\\
      \hline
    \end{tabular}
  \end{center}
\end{table}

\subsection{Uma subseção}

Bla blabla blablabla bla.  Bla blabla blablabla bla.  Bla blabla
blablabla bla.  Bla blabla blablabla bla.  Bla blabla blablabla bla.
Bla blabla blablabla bla.  Bla blabla blablabla bla.  Bla blabla
blablabla bla.  Bla blabla blablabla bla.  Bla blabla blablabla bla.
Bla blabla blablabla bla.  Bla blabla blablabla bla.  Bla blabla
blablabla bla.  Bla blabla blablabla bla.  Bla blabla blablabla bla.
Bla blabla blablabla bla.  Bla blabla blablabla bla.  Bla blabla
blablabla bla.  Bla blabla blablabla bla.  Bla blabla blablabla bla.
Bla blabla blablabla bla.

\chapter{Desenvolvimento}

  Bla blabla blablabla bla.  Bla blabla blablabla bla.  Bla blabla
  blablabla bla.  Bla blabla blablabla bla.  Bla blabla blablabla bla.
  Bla blabla blablabla bla.  Bla blabla blablabla bla.  Bla blabla
  blablabla bla.  Bla blabla blablabla bla.  Bla blabla blablabla bla.
  Bla blabla blablabla bla.  Bla blabla blablabla bla.  Bla blabla
  blablabla bla.  Bla blabla blablabla bla.  Bla blabla blablabla bla.
  Bla blabla blablabla bla.  Bla blabla blablabla bla.  Bla blabla
  blablabla bla.  Bla blabla blablabla bla.  Bla blabla blablabla bla.
  Bla blabla blablabla bla~\ref{tabela2}.

\begin{table}
\begin{center}
\caption{Nome da Tabela}\label{tabela2}
\begin{tabular}{p{4cm}p{5cm}p{6cm}}
\hline
Blabla & Blabla & Blablabla\\
\hline
{\small Bla} & {\small Blabla} & {\small\em Bla blabla blablabla blabla
  blablabla blabla blablabla.}\\
{\small Bla} & {\small Blabla} & {\small\em Bla blabla blablabla blabla
  blablabla blabla blablabla.}\\
{\small Bla} & {\small Blabla} & {\small\em Bla blabla blablabla blabla
  blablabla blabla blablabla.}\\
{\small Bla} & {\small Blabla} & {\small\em Bla blabla blablabla blabla
  blablabla blabla blablabla.}\\
{\small Bla} & {\small Blabla} & {\small\em Bla blabla blablabla blabla
  blablabla blabla blablabla.}\\
{\small Bla} & {\small Blabla} & {\small\em Bla blabla blablabla blabla
  blablabla blabla blablabla. Conforme a figura~\ref{figura}}\\
\hline
\end{tabular}
\end{center}
\end{table}

\begin{figure}[htbp]
  \centering \includegraphics[scale=.4]{figura}
\caption{Nome da figura} 
\label{figura}
\end{figure}


\chapter{Conclusão}

Bla blabla blablabla bla.  Bla blabla blablabla bla.  Bla blabla
blablabla bla.  Bla blabla blablabla bla.  Bla blabla blablabla bla.
Bla blabla blablabla bla.  Bla blabla blablabla bla.  Bla blabla
blablabla bla.  Bla blabla blablabla bla.  Bla blabla blablabla bla.
Bla blabla blablabla bla.  Bla blabla blablabla bla.  Bla blabla
blablabla bla.  Bla blabla blablabla bla.  Bla blabla blablabla bla.
Bla blabla blablabla bla.  Bla blabla blablabla bla.  Bla blabla
blablabla bla.  Bla blabla blablabla bla.  Bla blabla blablabla bla.
Bla blabla blablabla bla.

Bla blabla blablabla bla.  Bla blabla blablabla bla.  Bla blabla
blablabla bla.  Bla blabla blablabla bla.  Bla blabla blablabla bla.
Bla blabla blablabla bla.  Bla blabla blablabla bla.  Bla blabla
blablabla bla.  Bla blabla blablabla bla.  Bla blabla blablabla bla.
Bla blabla blablabla bla.  Bla blabla blablabla bla.  Bla blabla
blablabla bla.  Bla blabla blablabla bla.  Bla blabla blablabla bla.
Bla blabla blablabla bla.  Bla blabla blablabla bla.  Bla blabla
blablabla bla.  Bla blabla blablabla bla.  Bla blabla blablabla bla.
Bla blabla blablabla bla.

% Bibliografia http://liinwww.ira.uka.de/bibliography/index.html um
% site que cataloga no formato bibtex a bibliografia em computacao
% \bibliography{nomedoarquivo.bib} (sem extensao)
% \bibliographystyle{formato.bst} (sem extensao)

\bibliographystyle{abnt}
\bibliography{MaiconSantosTese2021} 

% % Apêndices (Opcional) - Material produzido pelo autor
% \apendices
% \chapter{Um Apêndice}

% % Anexos (Opcional) - Material produzido por outro
% \anexos
% \chapter{Um Anexo}

% Bla blabla blablabla bla.  Bla blabla blablabla bla.  Bla blabla
% blablabla bla.  Bla blabla blablabla bla.  Bla blabla blablabla bla.
% Bla blabla blablabla bla.  Bla blabla blablabla bla.  Bla blabla
% blablabla bla.  Bla blabla blablabla bla.  Bla blabla blablabla bla.
% Bla blabla blablabla bla.  Bla blabla blablabla bla.  Bla blabla
% blablabla bla.  Bla blabla blablabla bla.  Bla blabla blablabla bla.
% Bla blabla blablabla bla.  Bla blabla blablabla bla.  Bla blabla
% blablabla bla.  Bla blabla blablabla bla.  Bla blabla blablabla bla.
% Bla blabla blablabla bla.

% Bla blabla blablabla bla.  Bla blabla blablabla bla.  Bla blabla
% blablabla bla.  Bla blabla blablabla bla.  Bla blabla blablabla bla.
% Bla blabla blablabla bla.  Bla blabla blablabla bla.  Bla blabla
% blablabla bla.  Bla blabla blablabla bla.  Bla blabla blablabla bla.
% Bla blabla blablabla bla.  Bla blabla blablabla bla.  Bla blabla
% blablabla bla.  Bla blabla blablabla bla.  Bla blabla blablabla bla.
% Bla blabla blablabla bla.  Bla blabla blablabla bla.  Bla blabla
% blablabla bla.  Bla blabla blablabla bla.  Bla blabla blablabla bla.
% Bla blabla blablabla bla.

% \chapter{Outro Anexo}

% Bla blabla blablabla bla.  Bla blabla blablabla bla.  Bla blabla
% blablabla bla.  Bla blabla blablabla bla.  Bla blabla blablabla bla.
% Bla blabla blablabla bla.  Bla blabla blablabla bla.  Bla blabla
% blablabla bla.  Bla blabla blablabla bla.  Bla blabla blablabla bla.
% Bla blabla blablabla bla.  Bla blabla blablabla bla.  Bla blabla
% blablabla bla.  Bla blabla blablabla bla.  Bla blabla blablabla bla.
% Bla blabla blablabla bla.  Bla blabla blablabla bla.  Bla blabla
% blablabla bla.  Bla blabla blablabla bla.  Bla blabla blablabla bla.
% Bla blabla blablabla bla.

% Bla blabla blablabla bla.  Bla blabla blablabla bla.  Bla blabla
% blablabla bla.  Bla blabla blablabla bla.  Bla blabla blablabla bla.
% Bla blabla blablabla bla.  Bla blabla blablabla bla.  Bla blabla
% blablabla bla.  Bla blabla blablabla bla.  Bla blabla blablabla bla.
% Bla blabla blablabla bla.  Bla blabla blablabla bla.  Bla blabla
% blablabla bla.  Bla blabla blablabla bla.  Bla blabla blablabla bla.
% Bla blabla blablabla bla.  Bla blabla blablabla bla.  Bla blabla
% blablabla bla.  Bla blabla blablabla bla.  Bla blabla blablabla bla.
% Bla blabla blablabla bla.

% Faz a capa do CDROM
% \makecover

\end{document}